\documentclass[12pt]{article}

\usepackage{amsmath}
\usepackage{graphicx}
\usepackage{hyperref}


\title{Summary report for Machine Learning}

\author{\textsc{Magdalena Pakuła} \& \textsc{Jakub Pawlak}}

\date{2024–12–10}

\begin{document}

\maketitle

\section{Introduction}

Here is the introduction of our report

\begin{enumerate}

	\item Jeden
	      
	\item Two
	      
	\item Three
	      
\end{enumerate}


\section{Collaborative filtering}



\subsection{Model}

We changed some of the notation from the lecture example, to try make it more intuitive what different variables mean.
This section serves to precisely formulate the meanings of values used in the later sections.

Users will be denoted by $u$, where $U$ is the total number of users.
Movies will be denoted by $m$, where $M$ is the total number of movies.

\subsubsection{Movie ratings}

Let $y_{m,u}$, or alternatively $y[m,u]$ denote the rating of user $m$ by user $u$.

Therefore, $y$ will be an $M \times U$ matrix:
$$ y =
	\begin{bmatrix}
		y_{1,1} & y_{1,2} & \cdots & y_{1,U} \\
		y_{2,1} & y_{2,2} & \cdots & y_{2,U} \\
		\vdots  & \vdots  & \ddots & \vdots  \\
		y_{M,1} & y_{M,2} & \cdots & y_{M,U}
	\end{bmatrix}
$$

Valid ratings are integers from 0 to 5.
If $y_{m,u}$ takes value of NaN, it will denote that user $u$ did not rate movie $m$.

\subsubsection{Movie features}

Let $N$ be the total number of features.
Then, $x_{m,n}$ will be the $n$-th feature of the movie $m$.

As such, $p$ will be an $M \times N$ matrix:
$$ x =
	\begin{bmatrix}
		x_{1,1} & x_{1,2} & \cdots & x_{1,N} \\
		x_{2,1} & x_{2,2} & \cdots & x_{2,N} \\
		\vdots  & \vdots  & \ddots & \vdots  \\
		x_{M,1} & x_{M,2} & \cdots & x_{M,N}
	\end{bmatrix}
$$

\subsubsection{User parameters}

Each user $u$ will have a set of parameters $p_u$ corresponding to the movie features,
plus one additional feature $p_{u,0}$

$$ p =
	\begin{bmatrix}
		p_{1,0} & p_{1,1} & p_{1,2} & \cdots & p_{1,N} \\
		p_{2,0} & p_{2,1} & p_{2,2} & \cdots & p_{2,N} \\
		\vdots  & \vdots  & \vdots  & \ddots & \vdots  \\
		p_{U,0} & p_{U,1} & p_{U,2} & \cdots & p_{U,N}
	\end{bmatrix}
$$

\subsection{Calculating predictions}

Predicted rating of movie $m$ by user $u$ will be denoted as $\hat{y}(m,u)$
It will be calculated in the following way:
$$
	\hat{y}(m,u) = p_{u,0} + \sum\limits_{n=1}^N p_{u,n} \cdot x_{m,n}
$$

Alternatively, as a dot product:
$$
	\hat{y}(m,u) = p_{u,0} + p[u,1\!:] \cdot x[m,:],
$$
where
$$
	p[u,1\!:] = \begin{bmatrix}
		p_{u,1} & p_{u,2} & \cdots & p_{u,N}
	\end{bmatrix},$$
$$
	x[m,:] = \begin{bmatrix}
		x_{1,1} & x_{1,2} & \cdots & x_{1,N}
	\end{bmatrix}.
$$
In code:

\subsection{Calculating errors}

The error function will be as follows:
$$
	Q(p,x) = \frac{1}{2} \sum\limits_{m,u : y[m,u] \neq -1} (\hat{y}(m,u) - y[m,u])^2
$$

\subsection{Partial derivatives}

\subsubsection{Zeroth user parameter}

$$
	\frac{\partial Q}{\partial p_{u, 0}} = \sum_{m : y[m,u] \neq -1} \big(\hat{y}[m, u] - y[m, u]\big)
$$

\subsubsection{Other user parameters (1, \dots, N)}

$$
	\frac{\partial Q}{\partial p_{u, n}} = \sum_{m : y[m,u] \neq -1} \big(\hat{y}[m, u] - y[m, u]\big) \cdot x[m, n]
$$

\subsubsection{Movie parameters}

$$
	\frac{\partial Q}{\partial x_{m, n}} = \sum_{u : y[m,u] \neq -1} \big(\hat{y}[m, u] - y[m, u]\big) \cdot p[u, n]
$$

\subsubsection{Performance improvements}

After implementing the initial approach, it quickly became apparent that calculation of errors and gradients inside of a loop was very inefficient.


\end{document}
